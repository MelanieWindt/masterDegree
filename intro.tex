\intro
С необходимостью учитывать процессы излучения часто сталкиваются при численном моделирования высокотемпературных процессов. Среди них задачи, связанные с моделированием звездных атмосфер и межзвездного вещества, динамикой высокотемпературной плазмы, проектированием теплозащиты спускаемых аппаратов. Современные постановки этих задач существенно сложны и опираются, как правило, на трехмерное пространственное описание процессов. В то же время, в этих задачах излучение может иметь значительную и, даже, определяющую роль.

Сама по себе задача нахождения поля излучения довольно сложна. В общем случае, интенсивность излучения является функцией координат, времени, направления и частоты излучения. Решение полной задачи — крайне трудоемкая задача. Но влияние излучения на другие процессы обычно описывается некоторыми интегральными характеристиками интенсивности излучения. Поэтому, на практике, вместо полной постановки используется некоторым способом осредненная (по частотам, направлениям) приближенная постановка, имеющая существенно меньшую размерность.

Развитие численных методов для решения уравнения переноса излучения проходило с 50-х годов прошлого столетия. В то время разрабатывались методы, учитывающие ту или иную симметрию задачи, сводящую ее к эффективно одномерной или двумерной. Однако, эти методы совершенно непригодны для современных задач, и не являются универсальными.