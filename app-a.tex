\chapter{Основные алгоритмы}

\begin{algorithm}[ht!]
\centering
\begin{algorithmic}[1]
\Function{Coloring}{$ $}
\State $Q := $ \Call{InitQueue}{$ $}
\State $cnt$ := $|Q|$
\While {$Q \neq \varnothing$}
\State $t$ := \Call{Dequeue}{$Q$}
\If {\Call{Colorable}{$Q$}}
\State $color[t] := \Call{ChooseColor}{t}$
\For{$t' \in \Call{UncoloredNeighbours}{t}$}
\State \Call{Enqueue}{$Q, t'$}
\EndFor
\State $cnt := |Q|$
\Else
\State \Call{Enqueue}{$Q, t$}
\State $cnt := cnt - 1$
\EndIf
\If{$cnt < 0$}
\State \textbf{error} Обнаружен цикл
\EndIf
\EndWhile
\EndFunction
\end{algorithmic}
\caption{}
\label{alg:1}
\end{algorithm}

Функция \textsc{InitQueue} помещает в очередь такие тетраэдры, у которых все входные грани являются граничными.

Функция \textsc{Colorable} проверяет, можно ли покрасить рассматриваемый тетраэдр, для этого необходимо, чтобы все тетраэдры, граничащие с ним по входным граням, были покрашены. Если какая-то входная грань является граничной, то считается, что цвет отсутствующего тетраэдра равен $-1$. Функция \textsc{ChooseColor} выбирает максимальное значение цвета таких граничащих тетраэдров и увеличивает его на единицу. Это число и будет цветом рассматриваемого тетраэдра. 

\begin{algorithm}[ht!]
\centering
\begin{algorithmic}[1]
\Function{IsOuter}{$\vec \Omega, face$}
\State $flippedFace := \Call{Flip}{face}$
\If {$index(face) > index(flippedFace)$}
\State \Return $\vec \Omega \cdot normal(face) < 0$
\Else 
\State\Return $\vec \Omega \cdot normal(flippedFace) \geqslant 0$
\EndIf
\EndFunction
\end{algorithmic}
\caption{Алгоритм определения выходной грани.}
\label{alg:2}
\end{algorithm}

Для того, чтобы в соседних тетраэдрах одна и та же грань однозначно определялась, как входная для одного и выходная для другого, при определении предпочтение отдается грани с б\'{о}льшим номером.
