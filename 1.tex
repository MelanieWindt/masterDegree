\chapter{Постановка задачи}

\section{Математическая модель}
Поле излучения, заполняющего пространство, описывается распределением интенсивности излучения по частотам, в пространстве и по направлениям переноса лучистой энергии. Пусть $f(\nu, \vec r, \vec \Omega, t)d\nu d\vec r d \vec \Omega \, $ есть число световых квантов в спектральном интервале от $ \nu$ до $ \nu + d\nu$, находящихся в момент $t$ в элементе объема $d\vec r$ около точки $\vec r$ и имеющих направление движения в элементе телесного угла $d\vec \Omega$ около единичного вектора $\vec \Omega$. 
Каждый квант обладает энергией $h \nu$ и движется со скоростью $c$, поэтому величина 
\begin {equation}
I_{\nu} (\vec r, \vec \Omega, t)d \nu d\vec \Omega = h\nu c f (\nu, \vec r, \vec \Omega, t)d\nu d\vec{\Omega}
\end {equation}
есть количество лучистой энергии в спектральном интервале $d\nu$, протекающей в $1 \text{ сек}$ через площадку в $1 \text{ см}^2$, помещенную в точке $\vec r$ перпендикулярно к направлениям распространения энергии, которые лежат в элементе телесного угла $d\vec\Omega$ около вектора $\vec \Omega$. Задание функций $I_{\nu}$ или $f$ полностью определяет поле излучения. Количество лучистой энергии, заключенной  спектральном интервале $d\nu$ и находящейся в $1 \text{ см}^3$ пространства в точке $\vec r$ в момент $t$, или спектральная плотность излучения, равно:
\begin {equation}
U_\nu (\vec r, t) = h \nu \int_{4 \pi} f d \Omega = \frac{1}{c} \int_{4\pi} I_{\nu} d\Omega.
\end {equation}

Вектор спектрального потока равен 
\begin {equation}
S_{\nu} = \int I_{\nu}\vec\Omega d\Omega,
\end {equation}
где $\vec\Omega$ - единичный вектор направления движения квантов. 
Полные интенсивность, плотность и поток излучения получаются из спектральных интегрированием их по всему спектру частот:
\begin {equation}
I = \int_0^\infty I_{\nu} d\nu, \quad U = \int_0^\infty U_{\nu}d\nu, \quad \vec S =  \int_0^\infty \vec S_{\nu}d\nu.
\end {equation}
\subsection{Уравнение переноса}
Спектральная функция плотности равновесного излучения $U_{\nu p}$ может быть получена с помощью квантовой статистики. Количество энергии равновесного излучения частоты $\nu$ в $1 \text{ см}^3$, приходящееся на единичный интервал частот, равно
\begin {equation}
U_{\nu p} = \frac{8 \pi h \nu^3}{c^3} \frac {1}{e^{\frac{h\nu}{kT}} - 1}
\end {equation}
В силу изотропии спектральная интенсивность равновесного излучения равна
\begin {equation}
I_{\nu p} = \frac{cU_{\nu p}}{4\pi} = \frac{2h\nu^3}{c^2}\frac{1}{e^{\frac{h\nu}{kT}} - 1}
\label{5}
\end {equation}

Рассмотрим баланс излучения в элементарном цилиндре с площадью основания $d\sigma$ и высотой $ds$, расположенном в данной точке пространства таким образом, что направление $\vec\Omega$ совпадает с образующей цилиндра и перпендикулярно к его основаниям. За время $dt$ в левое основание втекает количество излучения $I_{\nu} (\vec\Omega, \vec r,t)d\sigma dt$. Из правого основания за тот же промежуток времени вытекает количество излучения $(I_{\nu} + dI_{\nu})d \sigma dt$.

Интенсивность $I_{\nu}$ есть функция координат и времени. Приращение интенсивности пучка света при переходе от левого основания к правому складывается из локального приращения за время прохождения светом пути $ds$ и из приращения при переходе от координаты $s$ к координате $s+ds$ в данный момент времени, 
\begin {equation}
dI_{\nu}  = \frac{\partial I_{\nu}}{\partial t } \frac{ds}{c}+\frac{\partial I_{\nu}}{\partial s}ds.
\end {equation}

Количество лучистой энергии в интервале частот $d\nu$ и интервале направлений $d\vec\Omega$, поглощаемой в $1 \text{ см}^3$ в  $1 \text{ сек}$, равно
\begin {equation}
I_{\nu}d\nu d\vec\Omega \varkappa_{\nu}.
\label {3}
\end {equation}

Количество энергии, самопроизвольно испускаемой веществом  в $1 \text{ см}^3$ в  $1 \text{ сек}$ интервале $d\nu d\vec\Omega$, равно $j_{\nu}d\nu d\vec\Omega$. Этим не исчерпывается полное количество излучения, испускаемого веществом. Существует так называемое вынужденное испускание. Вероятность вынужденного испускания кванта данной частоты и данного направления пропорциональна имеющейся в данной точке пространства интенсивности излучения той же частоты и того же направления. В квантовой теории показывается, что полная вероятность испускания данных квантов пропорциональна величине $1+n$, где $n$ - число фотонов с определенным направлением поляризации, находящихся в той же фазовой ячейке, в которую попадает испущенный квант. Это число равно $n= c^2I_{\nu}/2h\nu^3$. Таким образом, полное количество излучения, испускаемого в $1 \text{ сек}$ в $1 \text{ см}^3$ в интервале $d\nu d\vec\Omega$, равно 
\begin {equation}
j_{\nu} d\nu d\vec\Omega(1+\frac{c^2}{2h\nu^3}I_{\nu}).
\label{4}
\end {equation}

Первое слагаемое в скобках соответствует спонтанному испусканию, а второе - вынужденному. В состоянии термодинамического равновесия испускание и поглощение квантов данных частоты и направления в точности компенсируют друг друга, так что выражения \eqref{3} и \eqref{4} следует приравнять, причем интенсивность излучения $I_{\nu}$ заменить при этом равновесной величиной $I_{\nu p}$. 

Принимая во внимание формулу \eqref{5} для равновесной интенсивности, найдем, что отношение лучеиспускательной способности любого вещества к его коэффициенту поглощения есть универсальная функция частоты и температуры:
\begin {equation}
\frac{j_{\nu}}{\varkappa_{\nu}} =  \frac{I_{\nu p}}{1+\frac{c^2}{2h\nu^3}I_{\nu p}} = \frac{2h\nu^3}{c^2}e^{-\frac{h\nu}{kT}}.
\label{6}
\end {equation}

Это отношение представляет собой одну из форм закона Кирхгофа. Формулу \eqref{6} удобно переписать в виде
\begin {equation}
j_{\nu} = I_{\nu p}\varkappa{\nu}(1-e^{-\frac{h\nu}{kT}}).
\label{7}
\end {equation}

Количество излучения, испущенного в цилиндре за время $dt$, согласно формуле \eqref{4}, равно
\begin {equation}
j_{\nu}(1 + \frac {c^2}{2h\nu^3}I_{\nu})d\sigma ds dt.
\end {equation}
Поглощается в нем за то же время количество излучения $\varkappa_{\nu} I_{\nu}d\sigma dsdt$. Составляя баланс и поделив полученное выражение на произведение дифференциалов $d\sigma dsdt$, получим уравнение
\begin {equation}
\frac{1}{c} (\frac{\partial I_{\nu}}{\partial t} + c \vec \Omega \nabla I_{\nu}) = j_{\nu} (1 + \frac {c^2}{2h\nu^3}I_{\nu}) - \varkappa_{\nu}I_{\nu}.
\label{1}
\end {equation}

Преобразуем правую часть уравнения \eqref{1}, объединив вместе члены, отвечающие поглощению и вынужденному испусканию, поскольку они оба пропорциональны неизвестной функции координат и времени - интенсивности излучения. Введем при этом в множитель перед $I_{\nu}$ в члене вынужденного испускания вместо коэффициента излучения $j_{\nu}$ его выражение через коэффициент поглощения, в которое подставим формулу \eqref{5} для равновесной интенсивности. Правая часть уравнения примет вид 
\begin {equation}
j_{\nu} - \varkappa_{\nu}(1 - e^{-\frac{h\nu}{kT}})I_{\nu}.
\end {equation}
Отсюда видно, что вынужденное испускание можно трактовать как некое уменьшение поглощения: часть квантов как бы поглощается и тут же испускается снова с той же частотой и в том же направлении, причем вероятность этого "переизлучения" равна $e^{-\frac{h\nu}{kT}}$. Физически такие акты "переизлучения" никак себя не проявляет и их можно вообще исключить из рассмотрения, если считать, что коэффициент поглощения имеет несколько меньшую величину:
\begin {equation}
\varkappa'_{\nu} = \varkappa_{\nu}(1 - e^{-\frac{h\nu}{kT}}).
\end {equation}

Взаимодействие излучения с веществом можно представлять так, как будто существует только спонтанное испускание и эффективное поглощение, описываемое коэффициентом $\varkappa'_{\nu}$, исправленным на вынужденное испускание.

В новой трактовке закон Кирхгофа \eqref{7} приобретает форму
\begin {equation}
j_{\nu} = \varkappa'_{\nu}I_{\nu p}, \quad \varkappa'_{\nu} = \varkappa_{\nu}(1 - e^{-\frac{h\nu}{kT}}).
\end {equation}

Вводя это выражение в правую часть уравнения переноса \eqref{1} запишем уравнение в следующей форме:
\begin {equation}
\frac{1}{c}\frac{\partial I_{\nu}}{\partial t} + (\vec\Omega \nabla) I_{\nu} = \varkappa'_{\nu} (I_{\nu p} - I_{\nu}).
\label{2}
\end {equation}
\subsection{Аналитическое решение уравнения переноса}
Найдем формальное решение уравнения переноса излучения, рассматривая величины, зависящие только от состояния вещества $I_{\nu p}(T)$, $\varkappa'_{\nu}(T, \rho)$, как известные функции координат и времени. Рассмотрим сначала для простоты стационарный случай, когда распределения температуры и плотности, а также поле излучения не зависят от времени. Будем интересоваться излучением в точке $\vec r$ тела с направлением распространения $\vec \Omega$. Проведем луч через данную точку в данном направлении и обозначим координату вдоль луча через $s$. Замечая, что дифференциальное выражение в левой части уравнения переноса \eqref{2} представляет собой полную производную от интенсивности данного пакета квантов вдоль луча их распространения, перепишем уравнение в виде
\begin {equation}
\frac{dI_{\nu}}{ds} + \varkappa'_{\nu}I_{\nu} = \varkappa'_{\nu}I_{\nu p}.
\end {equation}
Это уравнение можно рассматривать как обыкновенное линейное уравнение относительно интенсивности вдоль луча. Решение его есть:
\begin {equation}
I_{\nu}(s) = \int_{s_0}^s\varkappa'_{\nu}I_{\nu p} exp\Big[-\int_{s'}^s\varkappa'_{\nu}ds''\Big]ds' + I_{\nu_0} exp\Big[-\int_{s_0}^s \varkappa'_{\nu}ds''\Big].
\end {equation}

Здесь $I_{\nu}(s)$ - интенсивность $I_{\nu}(\vec r, \vec\Omega)$, которая рассматривается как функция координаты $s$ вдоль луча. Интегрирование по лучу ведется, вообще говоря, от $-\infty$, а фактически от границы тела $s_0$ (как показано на рис.). Через $I_{\nu_ 0}$ обозначена константа интегрирования. 

