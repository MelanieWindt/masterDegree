\conclusion
Построен маршевый вычислительный алгоритм для решения стационарной задачи переноса излучения на неструктурированной трехмерной сетке во многогрупповом приближении. В основе маршевого метода лежит алгоритм упорядочения тетраэдров. В данной работе построен более универсальный алгоритм упорядочения тетраэдров, чем описанный в \cite{skalko_2014}, так как работает для более широкого класса триангуляций, чем триангуляции, удовлетворяющие условию Делоне.

Было найдено необходимое условие устойчивости данного численного метода, применительно к модельной задаче на равномерной сетке. Был реализован метод второго порядка аппроксимации и его монотонная модификация. 

Проведено сравнение методов на различных модельных задачах. Показано, что немонотонный вариант метода второго порядка может иметь нефизические осцилляции величиной порядка $20 \%$. Продемонстрировано различное качественное поведение численной диффузии излучения для методов первого и второго порядков аппроксимации.

Вычислительный алгоритм был применен к прикладной задаче воспроизведения спектра излучения звезды по результатам МГД моделирования взаимодействия магнитного поля звезды с веществом аккреционного диска. 