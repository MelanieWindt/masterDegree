\chapter{Численный метод}

Численный метод. Выбирается фиксированный набор направлений. В этих направлениях решается уравнение переноса. Решение строится маршевым методом (за один проход по области, нет итераций). Интегральные характеристики излучения (плотность и поток излучения) рассчитываются по квадратурной формуле, согласованной с набором направлений. Для решения задачи в одном направлении используется метод, использующий короткие характеристики. Все грани тетраэдров делятся на входящие и
выходящие. На выходящих гранях берется несколько точек, из которых выпускаются характеристики до пересечения с входными гранями. Вдоль каждой короткой характеристики решение строится по формуле 
\begin {equation}
I = I^*e^{\varkappa \lambda)} + I_p (1 - e^{-\varkappa \lambda})
\end {equation}
Постоянное значение решения на грани. Необходимо выпустить 1 характеристику. Линейное распределение решения на грани. Необходимо выпустить 3 характеристики

\section{Общие положения метода коротких характеристик}
\section{Подробное описание метода коротких характеристик (входные/выходные грани...)}
\section{Маршевый метод (упорядочивание, Делоне, алгоритм для не-Делоне)}
Для того, чтобы посчитать решение на выходящих гранях, нужно упорядочить тетраэдры. Это возможно, если нет циклов. В сетках, удовлетворяющих условию Делоне, циклов гарантированно нет. Упорядочивание тетраэдров производится алгоритмом окрашивания, аналогичным поиску в ширину. Используется очередь тетраэдров, в которую тетраэдр попадает при окрашивании одной из входящих граней, а удаляется, когда все его входящие грани окрашены.
\section{Особенности выбора точек на грани (случай 1D, когда схема не устойчива)}
В одномерном случае на равномерной сетке видны проблемы, если узлы брать строго внутри граней. При малых числах Куранта характеристики остаются в пределах того же элемента, и численное решение не сходится к точному (численное решение меняется только в одной ячейке, в то время как точное должно переноситься) На неструктурированной сетке этот эффект не проявляется, но, возможно, снижает порядок метода.
\section{Повышение порядка пространственной аппроксимации. Монотонная интерполяция}
Для метода первого порядка узлы взяты в вершинах граней. Решение на грани является линейной функцией координат. Схема всегда монотонна

Для метода второго порядка узлы взяты в вершинах и на ребрах. Решение на грани является квадратичной функцией координат. Схема немонотонна, чтобы обеспечить монотонность нужно использовать ограничитель. Для обеспечения монотонности достаточно, чтобы квадратичная интерполяция на каждом ребре была монотонной Квадратичная интерполяция по трем точкам будет монотонной, если выполнено соотношение (). Если соотношение нарушается, значение в центре корректируется