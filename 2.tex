\chapter{Численный метод}

\section{Общие положения метода коротких характеристик}
Уравнение \eqref{2} при каждом $\nu$ имеет гиперболический тип и семейство характеристик, которые просто являются характеристиками одномерных уравнений переноса вдоль направления $\vec\Omega$ 
\begin {equation}
\vec r - \vec r_0 = \vec \Omega c(t-t_0).
\end {equation}

Рассмотрим нестационарное уравнение переноса
\begin {equation}
\frac{1}{c}\frac{\partial I_{\nu}}{\partial t} + (\vec\Omega \nabla) I_{\nu} + \varkappa_\nu I_\nu = \varkappa_{\nu} I_{\nu p}.
\end {equation}

Выберем некоторым образом на единичной сфере набор направлений $\Theta = \{\vec\omega_i\}_1^n$. Для каждого направления уравнение является одномерным уравнением переноса, причем между собой уравнения не связаны:
\begin {equation}
\begin {cases}
\dfrac{1}{c} \dfrac{\partial I_{\nu,1}}{\partial t} + (\vec\omega_1 \nabla) I_{\nu,1} + \varkappa_\nu I_{\nu,1} = \varkappa_{\nu} I_{\nu p}, \\[12pt]
\dfrac{1}{c} \dfrac{\partial I_{\nu,2}}{\partial t} + (\vec\omega_2 \nabla) I_{\nu,2} + \varkappa_\nu I_{\nu,2} = \varkappa_{\nu} I_{\nu p},\\
\hspace{7,65em}\vdots \\
\dfrac{1}{c} \dfrac{\partial I_{\nu,n}}{\partial t} + (\vec\omega_n \nabla) I_{\nu,n} + \varkappa_\nu I_{\nu,n} = \varkappa_{\nu} I_{\nu p}.
\label {8}
\end {cases}
\end {equation}
С набором направлений $\Theta$ можно связать квадратурную формулу для сферы:
\begin {equation}
\int f(\vec\Omega)d\Omega \approx \sum_{i=1}^n w_i f(\vec \omega_i)
\end {equation}
Соответственно, моменты интенсивности $U, \vec S$ можно определить как
\begin {equation}
U = \sum_{i=1}^n w_i I_i, \quad
\vec S = \sum_{i=1}^n w_i \vec \omega_i I_i.
\end {equation}

Неопределенность касается выбора направлений $\Theta$ и численного метода решения каждого уравнения \eqref{8}.

\section{Подробное описание метода коротких характеристик}
Предположим, что в области решения задачи построена тетраэдральная сетка.
Пусть неизвестная функция интенсивности задана на гранях тетраэдра. В каждом тетраэдре решается уравнение переноса вдоль характеристики точно в предположении, что свойства вещества в тетраэдре постоянны. 

Каждое из уравнений \eqref{8} является гиперболическим и для его решения можно применить сеточно-характеристический метод. Для вычисления интенсивности $I_{\nu,i}$ в точке $p$ выпускается луч в направлении $-\vec\omega_i$ до пересечения с первой гранью. В точке пересечения $\vec r^*$ вычисляется интерполированное по вершинам грани значение интенсивности $I_{\nu, i} (\vec r^*) = \sum_{j = q,r,s} \alpha_jI_{\nu, i}(\vec r_j)$
(картинка). Далее, $I_{\nu, i} (\vec r_p)$ вычисляется из решения одномерного уравнения переноса   
\begin {equation}
I_{\nu}(s) = \int_{s_0}^s\varkappa'_{\nu}I_{\nu p} \exp\Big[-\int_{s'}^s\varkappa'_{\nu}ds''\Big]ds' + I_{\nu,0} \exp\Big[-\int_{s_0}^s \varkappa'_{\nu}ds''\Big].
\end {equation}

В простом случае, если в тетраэдре $\varkappa_\nu = \operatorname{const}$, формула упрощается до 
\begin {equation}
I_\nu (s) = I_{\nu p} (1 - e^{-\varkappa_\nu \Delta}) + I_{\nu,0}e^{-\varkappa_\nu \Delta},
\end {equation}
где $\Delta = s - s_0$. Очевидно, что $I_\nu(s)$ лежит в пределах от $I_{\nu, 0}$ до $I_{\nu p}$
\section{Маршевый метод}
Разделим грани тетраэдра на входящие и выходящие. Грань называется входящей, если луч, пересекающий ее, входит в тетраэдр, в противном случае~--- выходящей. Решение на выходящих гранях тетраэдра можно найти только тогда, когда решение на входящих гранях уже посчитано. Соответственно, на гранях триангуляции образуется отношение частичного порядка. В случае, если это отношение не образует цикл, его можно распространить на все множество граней, то есть задать такой порядок обхода граней, при котором выходные грани каждого тетраэдра идут после входных граней этого же тетраэдра. 

Для триангуляции Делоне порядок, в котором свет проходит грани такой же, в котором свет проходит центры описанных вокруг тетраэдра сфер, и требуемый порядок можно получить сортировкой координат центров тетраэдров вдоль направления переноса излучения \cite[] (Цитата на статью Скалько).

В случае, если триангуляция не является триангуляцией Делоне, такая сортировка может привести к неверному результату: могут образоваться тетраэдры, в которых выходные грани проходятся раньше входных. Однако задача упорядочения граней решается для более широкого класса, чем класс триангуляций Делоне. 

Предлагается следующий алгоритм упорядочения, являющийся некой адаптацией поиска в ширину. Частичный порядок граней порождает отношение порядка тетраэдров. Нужно пометить тетраэдры цветом (номером) так, чтобы отношение порядка было связано соответственно с возрастанием номера тетраэдра. Пусть $c(T)$ - номер (цвет) тетраэдра $T$. Тогда условие упорядоченности можно записать как $ c(T) > c (T')$ для всех $T'$, граничащих с $T$ по входной грани. 

Есть очередь тетраэдров, из которой извлекаются и в которую добавляются некоторые тетраэдры. В очереди сначала находятся все тетраэдры, грани которых были освещены. За шаг алгоритма из очереди извлекается один тетраэдр, проверяется, все ли его входящие грани освещены. В этом случае тетраэдр удаляется из очереди, а в очередь добавляются все его неосвещенные соседи. Если это не так, тетраэдр добавляется в конец очереди. При работе с очередью производится проверка на цикл: если обнаруживается, что из очереди нельзя удалить ни один тетраэдр, значит, у нас есть цикл, и алгоритм останавливается (дальнейшее использование маршевого метода невозможно).

Кроме самого упорядочения этот алгоритм группирует грани, решения на которых могут быть вычислены параллельны в силу своей независимости друг от друга. Это может стать основой для последующего распараллеливания алгоритма.
\section{Особенности выбора точек на грани}
Точки, из которых выпускаются характеристики на грани, не могут быть взяты произвольно. Рассмотрим данный метод, примененный к одномерному не стационарному уравнению переноса.

Когда число Куранта $\sigma < \sigma_\text{кр}$, схема теряет устойчивость, так как решение не может покинуть ячейку. Такого эффекта нет, если характеристики выпускаются из узлов расчетной сетки. Он проявляется на прямоугольных сетках, неизвестно, какой эффект он окажет на неструктурированные. В связи с этим в используемом методе в набор точек, из которых испускается характеристики, всегда включаются вершины граней. 
\section{Повышение порядка пространственной аппроксимации. Монотонная интерполяция}
В случае использования минимального количества (трех вершин на каждой грани), получается метод первого порядка, который имеет существенную численную диффузию луча. Получается схема первого порядка, чтобы увеличить точность схемы, используются дополнительные точки на гранях. В качестве таких точек выбираются середины ребер. 

Использование интерполяции на гранях по шести точкам позволяет поднять порядок аппроксимации метода коротких характеристик до второго. Такая параболическая интерпретация имеет существенный недостаток: она не является монотонной, то есть потенциально может приводить к появлению отрицательных значений интенсивности, что лишено физического смысла. Для того, чтобы избежать немонотонной интерполяции, нужно использовать процедуру ограничения решения в серединах ребер.
(картинка)
В численном методе значение интенсивности на ребрах ограничиваются по ближайшей границе допустимого интервала. Для выявления разницы между схемами первого и второго порядка аппроксимации использовались также кусочно-линейная интерполяция интенсивности по шести точкам. Что позволяет сравнить метод первого и второго порядка при одинаковом количестве (степеней свободы) узловых точек. 

В качестве базиса для квадратичной интерполяции используются функции $l0 - l5$, которые в стандартном треугольнике (единичном) имеют вид:

\begin {equation}
\begin {aligned}
\ell_0 &= (\eta + \xi -1)(2\eta + 2\xi - 1) \\
\ell_1 &= \xi(2\xi - 1) \\
\ell_2 &= \eta (2\eta - 1) \\
\ell_3 &= 4\eta\xi \\
\ell_4 &= -4\eta(\eta + \xi - 1) \\
\ell_5 &= -4\xi(\eta_\xi - 1) \\
\end {aligned}
\end {equation}

Для интерполяции первого порядка для тех же шести точек используются следующие функции Лагранжа:

\begin {equation}
\begin {aligned}
\ell_0 &= (1 - \xi - \eta) \\
\ell_1 &= \xi \\
\ell_2 &= \eta \\
\ell_3 &= 0 \\
\ell_4 &= 0 \\
\ell_5 &= 0 \\
\end {aligned}
\end {equation}

Причем, для случая первого порядка коррекция значений на ребрах не требуется.