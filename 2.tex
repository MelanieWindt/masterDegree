\chapter{Численный метод}

\section{Общие положения метода коротких характеристик}
Строим метод на тетраэдральной сетке, поэтому все грани делятся на входящие и выходящие, решение "привязано" к граням. В каждом тетраэдре решается уравнение переноса вдоль характеристики точно в предположении,что тетраэдр однороден. 

Каждый треугольник является входящей гранью для одного и выходящей - для другого (кроме граничных). 
\section{Подробное описание метода коротких характеристик (входные/выходные грани...)}
\section{Маршевый метод (упорядочивание, Делоне, алгоритм для не-Делоне)}
Нужно упорядочить грани таким образом, чтобы мы заполняли их в том же порядке, что и свет. Такое упорядочение не единственно, поэтому можно разбить грани на слои и "освещать" слои параллельно. \cite[]
(Цитата на статью Скалько).
Для триангуляции Делоне порядок, в котором свет проходит грани такой же, в котором свет проходит центры описанных вокруг тетраэдра сфер, и требуемый порядок можно получить сортировкой координат центров тетраэдров вдоль направления переноса излучения.

В случае, если триангуляция не является триангуляцией Делоне, такая сортировка делает неверный результат: свет проходит эти грани не в том порядке, в котором они отсортированы. Однако задача упорядочения граней решается для более широкого класса, чем триангуляция Делоне. 

Предлагается следующий алгоритм упорядочения, являющийся некой адаптацией поиска в ширину. Есть некий порядок для тетраэдров. Они могут быть упорядочены, если нет цикла, что верно для триангуляции Делоне и для более широкого класса триангуляций. Одна грань считается следующей за другой, если они находятся в одном тетраэре, причем одна является входящей, а другая - выходящей для этого тетраэдра. Нужно пометить тетраэдры цветом (номером) так, чтобы отношение порядка было связано соответственно с возрастанием номера тетраэдра. 

Есть очередь тетраэдров, из которой извлекаются и в которую добавляются некоторые тетраэдры. В очереди сначала находятся все тетраэдры, грани которых были освещены. За шаг алгоритма из очереди извлекается один тетраэдр, проверяется, все ли его входящие грани освещены. В этом случае тетраэдр удаляется из очереди, а в очередь добавляются все его неосвещенные соседи. Если это не так, тетраэдр добавляется в конец очереди. При работе с очередью производится проверка на цикл: если обнаруживается, что из очереди нельзя удалить ни один тетраэдр, значит, у нас есть цикл, и алгоритм останавливается (дальнейшее использование маршевого метода невозможно).

Кроме самого упорядочения этот алгоритм группирует грани, решения на которых могут быть вычислены параллельны в силу своей независимости друг от друга. Это может стать основой для последующего распараллеливания алгоритма.
\section{Особенности выбора точек на грани (случай 1D, когда схема не устойчива)}
Точки, из которых выпускаются характеристики на грани, не могут быть взяты произвольно. Рассмотрим данный метод, примененный к одномерному не стационарному уравнению переноса.

Когда число Куранта $\sigma < \sigma_{cr}$, схема теряет устойчивость, так как решение не может покинуть ячейку. Такого эффекта нет, если характеристики выпускаются из узлов расчетной сетки. Он проявляется на прямоугольных сетках, неизвестно, какой эффект он окажет на неструктурированные. В связи с этим в используемом методе в набор точек, из которых испускается характеристики, всегда включаются вершины граней. 
\section{Повышение порядка пространственной аппроксимации. Монотонная интерполяция}
В случае использования минимального количества (трех вершин на каждой грани), получается метод первого порядка, который имеет существенную численную диффузию луча. Получается схема первого порядка, чтобы увеличить точность схемы, используются дополнительные точки на гранях. В качестве таких точек выбираются середины ребер. 

Использование интерполяции на гранях по шести точкам позволяет поднять порядок аппроксимации метода коротких характеристик до второго. Такая параболическая интерпретация имеет существенный недостаток: она не является монотонной, то есть потенциально может приводить к появлению отрицательных значений интенсивности, что лишено физического смысла. Для того, чтобы избежать немонотонной интерполяции, нужно использовать процедуру ограничения решения в серединах ребер.
(картинка)
В численном методе значение интенсивности на ребрах ограничиваются по ближайшей границе допустимого интервала. Для выявления разницы между схемами первого и второго порядка аппроксимации использовались также кусочно-линейная интерполяция интенсивности по шести точкам. Что позволяет сравнить метод первого и второго порядка при одинаковом количестве (степеней свободы) узловых точек. 

В качестве базиса для квадратичной интерполяции используются функции $l0 - l5$, которые в стандартном треугольнике (единичном) имеют вид:

\begin {equation}
\begin {aligned}
\ell_0 &= (\eta + \xi -1)(2\eta + 2\xi - 1) \\
\ell_1 &= \xi(2\xi - 1) \\
\ell_2 &= \eta (2\eta - 1) \\
\ell_3 &= 4\eta\xi \\
\ell_4 &= -4\eta(\eta + \xi - 1) \\
\ell_5 &= -4\xi(\eta_\xi - 1) \\
\end {aligned}
\end {equation}

Для интерполяции первого порядка для тех же шести точек используются следующие функции Лагранжа:

\begin {equation}
\begin {aligned}
\ell_0 &= (1 - \xi - \eta) \\
\ell_1 &= \xi \\
\ell_2 &= \eta \\
\ell_3 &= 0 \\
\ell_4 &= 0 \\
\ell_5 &= 0 \\
\end {aligned}
\end {equation}

Причем, для случая первого порядка коррекция значений на ребрах не требуется.