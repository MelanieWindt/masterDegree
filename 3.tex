\chapter{Результаты}

\section{Сравнение методов первого и второго порядка}
\subsection{Решение уравнения в одном направлении}
Сравнение методов проводилось на следующей задаче. Рассматривалась
геометрическая область, куб со стороной $a = 2$. В центре области находится
крест, состоящий из пяти одинаковых кубиков со стороной $b = 0.2$ (см. рис. \ref{fig:6}).
\begin{figure}[ht!]
\centering{
\includegraphics[width = 0.5\textwidth]{cross.png}
}
\caption{Расчетная область}
\label{fig:6}
\end{figure}
Коэффи\-циент поглощения в области равен $\varkappa_1=0$, а внутри креста --- $\varkappa_2 = 100$. Равновесная интенсивность в центральной области $1$, а в окружающей среде --- $0$. Решение строилось вдоль одного направления, $\omega = (0,0,1)$ на сетке с $415625$ тетраэдрами и $76247$ вершинами. Изучалось решение на выходящей грани куба. 
\begin{figure}[ht!]
\centering{
\includegraphics[width = 0.3\textwidth]{1ord.png}%
\includegraphics[width = 0.3\textwidth]{2nolim.png}%
\includegraphics[width = 0.3\textwidth]{2wilim.png}
}
\caption{Сравнение решений методами первого порядка (слева), второго (в центре) и второго с ограничителем(справа).}
\label{fig:7}
\end{figure}

Точное решение должно представлять собой крест с интенсивностью $1 -
e^{b\varkappa_2} = 1-e^{-20} \approx 1$. Мы можем видеть (см. рис. \ref{fig:7}), что для первого порядка диффузия достаточно велика, второй порядок без ограничителя отклоняется от допустимых пределов $I \in [0, 1]$ на $20 \%$.
\subsection{Сравнение плотности излучения}
На той же самой задаче изучалась плотность излучения в центральном сечении куба, но коэффициенты поглощения изменились следующим образом: $\varkappa_1 = 10$, $\varkappa_2 = 1$. . Использовались 170 направлений из квадратурной формулы Лебедева \cite{lebedev_1999}. 

\begin{figure}[ht!]
\centering{
\includegraphics[width = \textwidth]{U2vs1.png}
}
\caption{Сравнение плотности излучения в центральном сечении куба. Слева второй порядок, справа --- первый.}
\label{fig:9}
\end{figure}

\begin{figure}[ht!]
\centering{
\includegraphics[width = \textwidth]{U2vs1Line.png}
}
\caption{Плотность излучения вдоль оси $Oz$. Красный второй порядок, синий - первый.}
\label{fig:10}
\end{figure}
Сравнение плотности излучения \ref{fig:9} и \ref{fig:10} показывает схожие результаты: интенсивность в центре куба в методе второго порядка на $\approx 2 \%$ больше, чем в случае метода первого порядка. В случае метода второго порядка \ref{fig:9} более выражен <<эффект луча>>, в то время как в методе первого порядка этот эффект сглажен за счет численной диффузии. 

\section{Сравнение методов первого и второго порядка с одинаковым количеством степеней свобод}
Рассматривалась геометрическая область: две концентрические сферы  с центром в точке $(0,0,0)$, внешняя радиусом $R = 5$, а внутренняя --- $r = 0.35$ (см. рис. \ref{fig:11}). Коэффициент поглощения между  сферами равен $\varkappa_1 = 0$, а внутри меньшей сферы --- $\varkappa_2 = 10$. Равновесная интенсивность в центральной области $1$, а в окружающей среде --- $0$. 
\begin{figure}[ht!]
\centering{
\includegraphics[width = 0.5\textwidth]{sphere.png}
}
\caption{Расчетная область в случае сравнения методов первого и второго порядка с одинаковым количеством степеней свобод.}
\label{fig:11}
\end{figure}
Решение строилось вдоль одного направления $\omega = (1, 0, 0)$ на сетке с $387736$ тетраэдрами и $70705$ точками. Изучалось решение в центральном сечении. 
\begin{figure}[ht!]
\centering{
\includegraphics[width = 0.7\textwidth]{2vs15.png}
}
\caption{Диффузия луча в методе первого (справа) и второго (слева) порядка при одинаковом количестве степеней свободы.}
\label{fig:12}
\end{figure}

Увеличение количества степеней свободы делает метод первого порядка практически таким же точным, как и метод второго порядка, однако численное рассеяние луча имеет различный характер в обоих случаях (см. рис. \ref{fig:12}). В случае метода второго порядка оно практически не меняется вдоль луча, в то время как для метода первого порядка оно растет пропорционально корню удаления от источника ($\Delta y^2 \sim x$), как и должно быть в случае метода первого порядка. 
\section{Более сложная задача из астрофизики}
